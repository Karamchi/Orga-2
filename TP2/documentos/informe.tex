\documentclass[a4paper]{article}
\usepackage[spanish]{babel}
\usepackage[utf8]{inputenc}
\usepackage{charter}   % tipografia
\usepackage{graphicx}
%\usepackage{makeidx}
\usepackage{paralist} %itemize inline

%\usepackage{float}
%\usepackage{amsmath, amsthm, amssymb}
%\usepackage{amsfonts}
%\usepackage{sectsty}
%\usepackage{charter}
%\usepackage{wrapfig}
%\usepackage{listings}
%\lstset{language=C}


\input{codesnippet}
\input{page.layout}
% \setcounter{secnumdepth}{2}
\usepackage{underscore}
\usepackage{caratula}
\usepackage{url}


% ******************************************************** %
%              TEMPLATE DE INFORME ORGA2 v0.1              %
% ******************************************************** %
% ******************************************************** %
%                                                          %
% ALGUNOS PAQUETES REQUERIDOS (EN UBUNTU):                 %
% ========================================
%                                                          %
% texlive-latex-base                                       %
% texlive-latex-recommended                                %
% texlive-fonts-recommended                                %
% texlive-latex-extra?                                     %
% texlive-lang-spanish (en ubuntu 13.10)                   %
% ******************************************************** %



\begin{document}


\thispagestyle{empty}
\materia{Organización del Computador II}
\submateria{Segundo Cuatrimestre de 2014}
\titulo{Trabajo Práctico II}
\subtitulo{subtitulo del trabajo}
\integrante{Cristian Chibana}{586/13}{christian.chiba93@gmail.com}
\integrante{Javier Minces Müller}{231/13}{javijavi1994@hotmail.com}
\integrante{Nicolás Roulet}{587/13}{nicoroulet@gmail.com}

\maketitle
\newpage

\thispagestyle{empty}
\vfill
\begin{abstract}
En este Trabajo Práctico se implementaron cuatro filtros de imagen en dos versiones: una en lenguaje C, y una en ASM haciendo uso de las instrucciones SSE, para realizar un análisis de la performance del procesador al utilizar el modelo de programación SIMD (Single Instruction Multiple Data). 

\end{abstract}

\thispagestyle{empty}
\vspace{3cm}
\tableofcontents
\newpage


%\normalsize
\newpage

\section{Objetivos generales}
(no sé qué más hacer que copiar un poco el enunciado)
Este Trabajo Práctico tiene como objetivos principales explorar el modelo de programación SIMD (Single Instruction Multiple Data) y realizar un análisis riguroso de los resultados de performance del procesador al hacer uso de las intrucciones SSE. Para esto, se implementaron cuatro filtros en dos versiones: una en lenguaje C, y una en ASM haciendo uso de las instrucciones SSE. 
Los filtros a aplicar son los siguientes:
* Cropflip: recorta una imagen y la voltea verticalmente, según cuatro parámetros que indican que fracción rectangular de la imágen se conserva.
(fórmula)
* Sierpinski: oscurece los puntos de la imagen, generando una forma similar a un triángulo de sierpinski
(fórmula)
* Bandas: convierte la imagen a x tonos de gris
(fórmula)
* Motion blur: aplica un filtro de desenfoque en movimiento, aplicando a cada pixel un promedio de algunos pixeles cercanos
(fórmula)

Los filtros son aplicables a imagenes de distintos tamaños, y también a videos.
Para cada experimento pensábamos realizar 10 mediciones, para descartar los outliers y calcular promedio y varianza 



\section{Contexto}

\begin{figure}
  \begin{center}
	\includegraphics[scale=0.66]{imagenes/logouba.jpg}
	\caption{Descripcion de la figura}
	\label{nombreparareferenciar}
  \end{center}
\end{figure}


\subsection{Filtro cropflip}

\subsection{Filtro sierpinski}
En cada iteración de la implementación en assembler del filtro sierpinski se procesan cuatro pixeles. Esto permite manejar cada pixel en un registro, si se toman sus valores de r, g, b y a como enteros de 4 bytes o como decimales de precisión simple (floats). En cada iteración, se desempaquetan los pixeles en cuatro registros xmm, extendiendo sus valores a 4 bytes. Luego calcula el valor del coeficiente de cada pixel en otro registro xmm, conservándolo como entero y realizando todos los productos antes que las divisiones para evitar perder precisión, sabiendo que nunca pueden excederse de 4 bytes. Para relizar las divisiones enteras, como no hay una instrucción que permita hacerlas, se convierten los valores del registro a float, se realiza la división y se convierten, truncando, a enteros de 4 bytes. Finalmente, se multiplica el valor de cada pixel por el coeficiente correspondiente, se empaqueta el resultado en bytes y se asigna a los pixeles.
{podemos poner pseudocódigo o imágenes, las imágenes venden, más allá de que terminen siendo scrinchots de calxcel}

Al comparar la versión en assembler con la versión en C con las distintas optimizaciones, obtuvimos estos resultados:

(gráfico)

Se puede ver que las optimizaciones de C {no} reducen apreciablemente el tiempo de ejecución. El tiempo de ejecución en C es alrededor del ¿triple? del tiempo de ejecución en assembler. [Esto es lógico pero hay que justificarlo].

Hallamos que el factor limitante era [la intensidad de cómputo/bus de memoria]. Se puede ver en los gráficos que los accesos a memoria {no} afectan más que las operaciones aritméticas.

(gráfico)

\subsection{Filtro bandas}
En cada iteración de la implementación en assembler del filtro bandas se procesan dos pixeles. 


\subsection{Filtro motion blur}



\paragraph{\textbf{Titulo del parrafo} } Bla bla bla bla.
Esto se muestra en la figura~\ref{nombreparareferenciar}.



\begin{codesnippet}
\begin{verbatim}

struct Pepe {

    ...

};

\end{verbatim}
\end{codesnippet}


\section{Enunciado y solucion} 
\input{enunciado}

\section{Conclusiones y trabajo futuro}


\end{document}

